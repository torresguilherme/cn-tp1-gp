\documentclass[11pt]{article}
\usepackage[utf8]{inputenc}
\usepackage[T1]{fontenc}
\usepackage{amssymb}
\usepackage{array}

\begin{document}

\title{Computação Natural - TP1 - Programação Genética}
\author{Guilherme Torres (Departamento de Ciência da Computação - UFMG)}
\date{}
\maketitle

\section{Introdução}

O objetivo deste trabalho foi formular e escolher os melhores parâmetros para um algoritmo de aprendizado para regressão simbólica. Dada uma função $f: \mathbb{R}^n \to \mathbb{R}$, a regressão simbólica consiste em tentar aproximar essa função usando uma árvore constituída por terminais nas folhas (constantes ou variáveis) e operadores.

A heurística usada foi a programação genética, que usa de conceitos evolucionários para evoluir as soluções para o problema e obter indivíduos mais adaptados. É um algoritmo estocástico, ou seja, com vários componentes de aleatoriedade como geração inicial dos indivíduos e mutações, o que requer que muitos experimentos sejam realizados para tirar as conclusões sobre o trabalho, que serão discutidos nesse documento.

O algoritmo foi implementado em Python 3, na pasta /src. Instruções para executar o teste estão explicitadas no arquivo README.md. O programa também pode ser encontrado no repositório:

\texttt{https://github.com/torresguilherme/cn-tp1-gp}.

\section{Implementação}

\subsection{Fundamentação}

O funcionamento do algoritmo pode ser descrito, no plano geral, pelos seguintes passos:

\end{document}